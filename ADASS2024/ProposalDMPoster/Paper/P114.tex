% This is the ADASS_template.tex LaTeX file, 19th Sep 2019.
% It is based on the ASP general author template file, but modified to reflect the specific
% requirements of the ADASS proceedings.
% Copyright 2014, Astronomical Society of the Pacific Conference Series
% Revision:  14 August 2014

% To compile, at the command line positioned at this folder, type:
% latex ADASS_template
% latex ADASS_template
% dvipdfm ADASS_template
% This will create a file called ADASS_template.pdf

\documentclass[11pt,twoside]{article}

% Do NOT use ANY packages other than asp2014. 
\usepackage{asp2014}

\aspSuppressVolSlug
\resetcounters

% References must all use BibTeX entries in a .bibfile.
% References must be cited in the text using \citet{} or \citep{}.
% Do not use \cite{}.
% See ManuscriptInstructions.pdf for more details
\bibliographystyle{asp2014}

% The ``markboth'' line sets up the running heads for the paper.
% 1 author: "Surname"
% 2 authors: "Surname1 and Surname2"
% 3 authors: "Surname1, Surname2, and Surname3"
% >3 authors: "Surname1 et al."
% Replace ``Short Title'' with the actual paper title, shortened if necessary.
% Use mixed case type for the shortened title
% Ensure shortened title does not cause an overfull hbox LaTeX error
% See ASPmanual2010.pdf 2.1.4  and ManuscriptInstructions.pdf for more details
\markboth{Paul Harrison}{Proposal Data Model}

\begin{document}

\title{The Proposal Data Model as the basis for Polaris, a Proposal Submission Toolkit}

% Note the position of the comma between the author name and the 
% affiliation number.
% Authors surnames should come after first names or initials, eg John Smith, or J. Smith.
% Author names should be separated by commas.
% The final author should be preceded by "and".
% Affiliations should not be repeated across multiple \affil commands. If several
% authors share an affiliation this should be in a single \affil which can then
% be referenced for several author names. If only one affiliation, no footnotes are needed.
% See ManuscriptInstructions.pdf and ASP's manual2010.pdf 3.1.4 for more details
\author{Paul~Harrison$^1$}
\affil{$^1$JBCA, University of Manchester,  UK; \email{paul.harrison@manchester.ac.uk}}


% This section is for ADS Processing.  There must be one line per author. paperauthor has 9 arguments.
\paperauthor{Paul~Harrison}{paul.harrison@manchester.ac.uk}{0000-0001-6362-3182}{University of Manchester}{JBCA}{Macclesfield}{State/Province}{SK11 9DL}{UK}

% There should be one \aindex line (commented out) for each author. These are used to
% build up the author index for the Proceedings. The surname must come first, followed by
% initials. Note the use of ~ before each initial to control spacing.
% The \author entries (see above) have surname last. These \aindex entries have
% surname first.
% The Aindex.py command willl create them for you after you have constructed the \author
% The first entry should be the first author, for bold-facing the author index page-reference

%\aindex{FistAuthor1,~S.~A.}
%\aindex{Author2,~S.~B.}
%\aindex{Author3,~S.}


\begin{abstract}
The Proposal Data Model (ProposalDM) has been created as an IVOA endorsed data model expressed in VO-DML to represent the metadata involved in creating and assessing
    Observing proposals for time to be allocated at astronomical facilities.

    The Polaris proposal toolkit is a set of services to allow proposals to be created and managed via web services APIs that can be accessed via a GUI or the
    commandline clients that are also part of the toolkit.
    The Polaris toolkit uses the ProposalDM at its native internal data model.
\end{abstract}

% These lines show examples of subject index entries. At this stage these have to commented
% out, and need to be on separate lines. Eventually, they will be automatically uncommented
% and used to generate entries in the Subject Index at the end of the Proceedings volume.
% Don't leave these in! - replace them with ones relevant to your paper.
%\ssindex{FOOBAR!conference!ADASS 2019}
%\ssindex{FOOBAR!organisations!ASP}

% These lines show examples of ASCL index entries. At this stage these have to commented
% out, and need to be on separate lines. Eventually, they will be automatically uncommented
% and used to generate entries in the ASCL Index at the end of the Proceedings volume.
% The ascl.py command will scan your paper on possible code names.
% Don't leave these in! - replace them with ones relevant to your paper.
%\ooindex{FOOBAR, ascl:1101.010}


\section{Motivation}
        Funding was obtained as part of the joint activity 2.1 of the \href{https://www.orp-h2020.eu}{Opticon RadioNet Pilot (ORP)} to create
        a new proposal preparation tool that would make it easier for astronomers to be able to participate in so-called
        multi-facility calls.
        Explicit in the funding request was the aim of creating a data model (in a similar fashion to\cite{10.1117/12.789262}) that could be used to describe the observation proposals
        in as general way as possible to cover a common set of requirements from as many different observatories and observing modes.
        That data model could  then be used as the basis for creating a toolkit for managing proposals.

        This poster describes the data model, called ProposalDM, and another poster \citep{p103_adassxxxiv} describes the toolkit, called Polaris.


        \subsection{Design Goals}
        The Data model is primarily intended to be able to support what is often called "phase 1" of the proposal process, i.e.\ the information
        in the model is sufficient to be able to allocate the observations, but not necessarily complete enough to actually schedule them.
        \begin{itemize}
            \item Emphasis on required physics and the science goals.
            \item Flexible enough to use at multiple observatories.
        \end{itemize}

    \section{Implementation}
        The model is expressed in VO-DML and published as an IVOA working draft. It has actually been split into two halves
        \begin{itemize}
            \item The proposal part, which expresses the concepts important to the astronomer principal investigator.
            \item The proposal management part, which expresses the concepts necessary to review and allocate the proposal.
        \end{itemize}
        The two figures show a somewhat simplified view of each of these parts, which despite this, are probably unintelligible on
        this poster!
        For an easier to read presentation of the model see \href{https://ivoa.github.io/ProposalDM/}{https://ivoa.github.io/ProposalDM/}.


    \articlefigure{proposaldm.vo-dml.eps}{pdmfig}{Simplified view of the Proposal part of the model.}



    \articlefigure{proposalManagement.vo-dml.eps}{manfig}{Simplified view of the Proposal Management part of the model.}


    \section{VO-DML}
        VO-DML \citep{2018ivoa.spec.0910L} is an IVOA standard for defining data models.
        It has been designed in a way that is more constrained and concise than other generic modelling methodologies such
        as UML, so that it is easy to create models with well defined serializations following IVOA standards.
        There is also some tooling associated with VO-DML (\href{https://github.com/ivoa/vo-dml}{https://github.com/ivoa/vo-dml}) that
        has the capability of generating documentation and source code data objects in Java and Python that allow
        easy serialization to and from XML and JSON as well as object relational mapping to databases.


    \section{Future plans}
        It should be noted that the ProposalDM has not reached the 1.0 status yet as there are still some parts of the model to elaborate.
        In addition it would probably be a good idea to factor out common Observatory, Instrument and Target models to allow re-use elsewhere in IVOA data models.

       The reader is encouraged to follow the web links within this document to learn the latest status of the projects.



\acknowledgements This project has received funding from the European Union's Horizon 2020 research and innovation programme under grant agreement No 101004719.


\bibliography{P114}  % For BibTex

% if we have space left, we might add a conference photograph here. Leave commented for now.
% \bookpartphoto[width=1.0\textwidth]{foobar.eps}{FooBar Photo (Photo: Any Photographer)}

    %\ssindex{proposal}

\end{document}
