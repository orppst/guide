
\documentclass[final]{beamer}

% ====================
% Packages orientation=portrait,
% ====================

\usepackage[T1]{fontenc}
\usepackage{lmodern}
\usepackage[size=custom,width=120,height=213,scale=2.5]{beamerposter}
\usetheme{gemini}
\usecolortheme{uom}
\usepackage{graphicx}
\graphicspath{{./images/}}
\usepackage{booktabs}
\usepackage{tikz}
\usepackage{pgfplots}
\pgfplotsset{compat=1.14}
\usepackage{anyfontsize}

% ====================
% Lengths
% ====================

% If you have N columns, choose \sepwidth and \colwidth such that
% (N+1)*\sepwidth + N*\colwidth = \paperwidth
\newlength{\sepwidth}
\newlength{\colwidth}
\setlength{\sepwidth}{0.025\paperwidth}
\setlength{\colwidth}{0.45\paperwidth}

\newcommand{\separatorcolumn}{\begin{column}{\sepwidth}\end{column}}

% ====================
% Title
% ====================

\title{ProposalDM}

\author{Paul Harrison}

\institute[shortinst]{JBCA, The University of Manchester, UK}

% ====================
% Footer Logos
% ====================
%logos left. middle, and right in the footer
\logoleftfooter{\includegraphics[height=7cm]{logos/IVOA_logo}}
\logomiddlefooter{\includegraphics[height=7cm]{logos/ADASS-XXXIV_Logo_rgb_FullColour_uwWfloC}}
\logorightfooter{\includegraphics[height=7cm]{images/ProposalDM_QR}}

\footercontent{
    \hspace{10ex} IVOA \hfill
  ADASS XXXIV Valetta, Malta - 2024 \hfill
    \href{https://github.com/ivoa/ProposalDM/}{github.com/ivoa/ProposalDM}{\color{manYellow}}
}


% ====================
% Header Logos
% ====================

% use this to include logos on the left and/or right side of the header:
\logoright{\includegraphics[height=7cm]{logos/INSU-Opticon-Radionet-logo-web-blanc-fond-transparent}}
\logoleft{\includegraphics[height=7cm]{logos/UOM_logo_allwhite}}

% ====================
% Figure command
% ====================

\newcommand{\insertFigure}[3][width=1.0\textwidth]{%
    \begin{figure}[ht]
    \centering
    \includegraphics[#1]{#2}
    \caption{#3}
    \label{fig:#2}
    \end{figure}
}


% ====================
% Body
% ====================

\begin{document}

    \begin{frame}[t]
        \begin{columns}[t]
            \separatorcolumn

            \begin{column}{\colwidth}

                \begin{block}{Motivation}
                    Funding was obtained as part of the joint activity 2.1 of the \href{https://www.orp-h2020.eu}{Opticon RadioNet Pilot (ORP)} to create
                    a new proposal preparation tool that would make it easier for astronomers to be able to participate in so-called
                    multi-facility calls.
                    Explicit in the funding request was the aim of creating a data model (in a similar fashion to\cite{10.1117/12.789262}) that could be used to describe the observation proposals
                    in as general way as possible to cover a common set of requirements from as many different observatories and observing modes.
                    That data model could  then be used as the basis for creating a toolkit for managing proposals.

                    This poster describes the data model, called ProposalDM, and another poster (P103) describes the toolkit, called Polaris.


                \heading{Design Goals}
                    The Data model is primarily intended to be able to support what is often called "phase 1" of the proposal process, i.e.\ the information
                    in the model is sufficient to be able to allocate the observations, but not necessarily complete enough to actually schedule them.
                    \begin{itemize}
                        \item Emphasis on required physics and the science goals.
                        \item Flexible enough to use at multiple observatories.
                    \end{itemize}
                \end{block}
                \begin{block}{Implementation}
                     The model is expressed in VO-DML and published as an IVOA working draft. It has actually been split into two halves
                    \begin{itemize}
                        \item The proposal part, which expresses the concepts important to the astronomer principal investigator.
                        \item The proposal management part, which expresses the concepts necessary to review and allocate the proposal.
                    \end{itemize}
                    The two figures show a somewhat simplified view of each of these parts, which despite this, are probably unintelligible on
                    this poster!
                    For an easier to read presentation of the model see \href{https://ivoa.github.io/ProposalDM/}{https://ivoa.github.io/ProposalDM/}.
                \end{block}

                \insertFigure{proposaldm.vo-dml}{Simplified view of the Proposal part of the model.}

            \end{column}

            \separatorcolumn

            \begin{column}{\colwidth}



                \insertFigure{proposalManagement.vo-dml}{Simplified view of the Proposal Management part of the model.}


                \begin{exampleblock}{VO-DML}
                  VO-DML\cite{2018ivoa.spec.0910L} is an IVOA standard for defining data models.
                  It has been designed in a way that is more constrained and concise than other generic modelling methodologies such
                    as UML, so that it is easy to create models with well defined serializations following IVOA standards.
                    There is also some tooling associated with VO-DML (\href{https://github.com/ivoa/vo-dml}{https://github.com/ivoa/vo-dml}) that
                    has the capability of generating documentation and source code data objects in Java and Python that allow
                    easy serialization to and from XML and JSON as well as object relational mapping to databases.
                \end{exampleblock}

                \begin{block}{Future plans}
                   It should be noted that the ProposalDM has not reached the 1.0 status yet as there are still some parts of the model to elaborate.
                   In addition it would probably be a good idea to factor out common Observatory, Instrument and Target models to allow re-use elsewhere in IVOA data models.

                   Any help/feedback from the community would be welcome.

                \end{block}

                \begin{block}{Acknowledgements}
                    \includegraphics[height=\baselineskip]{logos/EU_flag_yellow_eps}
                    This project has received funding from the \textbf{European Union’s Horizon 2020} research and
                    innovation programme under grant agreement \textbf{No 101004719}

                \end{block}


                \begin{block}{References}
                    \footnotesize{\bibliographystyle{plain}\bibliography{poster}}

                \end{block}

            \end{column}
            \separatorcolumn
        \end{columns}

    \end{frame}
\end{document}
