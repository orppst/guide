% This is the ADASS_template.tex LaTeX file, 19th Sep 2019.
% It is based on the ASP general author template file, but modified to reflect the specific
% requirements of the ADASS proceedings.
% Copyright 2014, Astronomical Society of the Pacific Conference Series
% Revision:  14 August 2014

% To compile, at the command line positioned at this folder, type:
% latex ADASS_template
% latex ADASS_template
% dvipdfm ADASS_template
% This will create a file called ADASS_template.pdf

\documentclass[11pt,twoside]{article}

% Do NOT use ANY packages other than asp2014. 
\usepackage{asp2014}

\aspSuppressVolSlug
\resetcounters

% References must all use BibTeX entries in a .bibfile.
% References must be cited in the text using \citet{} or \citep{}.
% Do not use \cite{}.
% See ManuscriptInstructions.pdf for more details
\bibliographystyle{asp2014}

% The ``markboth'' line sets up the running heads for the paper.
% 1 author: "Surname"
% 2 authors: "Surname1 and Surname2"
% 3 authors: "Surname1, Surname2, and Surname3"
% >3 authors: "Surname1 et al."
% Replace ``Short Title'' with the actual paper title, shortened if necessary.
% Use mixed case type for the shortened title
% Ensure shortened title does not cause an overfull hbox LaTeX error
% See ASPmanual2010.pdf 2.1.4  and ManuscriptInstructions.pdf for more details
\markboth{Walker et al}{Polaris: a new open source observation proposal preparation tool}

\begin{document}

\title{Polaris: A New Open Source Observation Proposal Preparation Tool}

% Note the position of the comma between the author name and the 
% affiliation number.
% Authors surnames should come after first names or initials, eg John Smith, or J. Smith.
% Author names should be separated by commas.
% The final author should be preceded by "and".
% Affiliations should not be repeated across multiple \affil commands. If several
% authors share an affiliation this should be in a single \affil which can then
% be referenced for several author names. If only one affiliation, no footnotes are needed.
% See ManuscriptInstructions.pdf and ASP's manual2010.pdf 3.1.4 for more details
\author{Darren~Walker, Allan~England, Ben~Green, and Paul~Harrison}
\affil{JCBA, The University of Manchester, Manchester, UK; \email{darren.walker@manchester.ac.uk}}

% This section is for ADS Processing.  There must be one line per author. paperauthor has 9 arguments.
% Arguments are: {NAME}{EMAIL}{ORCID_ID_OR_BLANK}{INSTITUTION}{DEPARTMENT}{CITY}{PROVINCE}{POSTAL CODE}{COUNTRY}
\paperauthor{Darren~Walker}{darren.walker@manchester.ac.uk}{}{The University of Manchester}{Dept. of Physics and Astronomy}{Jodrell Bank Observatory}{Cheshire}{SK11 9DL}{UK}
\paperauthor{Allan~England}{allan.england@manchester.ac.uk}{}{The University of Manchester}{Dept. of Physics and Astronomy}{Jodrell Bank Observatory}{Cheshire}{SK11 9DL}{UK}
\paperauthor{Ben~Green}{benjamin.green@manchester.ac.uk}{}{The University of Manchester}{Dept. of Physics and Astronomy}{Jodrell Bank Observatory}{Cheshire}{SK11 9DL}{UK}
\paperauthor{Paul~Harrison}{paul.harrison@manchester.ac.uk}{}{The University of Manchester}{Dept. of Physics and Astronomy}{Jodrell Bank Observatory}{Cheshire}{SK11 9DL}{UK}

% There should be one \aindex line (commented out) for each author. These are used to
% build up the author index for the Proceedings. The surname must come first, followed by
% initials. Note the use of ~ before each initial to control spacing.
% The \author entries (see above) have surname last. These \aindex entries have
% surname first.
% The Aindex.py command willl create them for you after you have constructed the \author
% The first entry should be the first author, for bold-facing the author index page-reference

%\aindex{Walker,~D.~J.}
%\aindex{England,~A.}
%\aindex{Green,~B.}
%\aindex{Harrison,~P.~A.}


\begin{abstract}
    We are developing a new, open source proposal tool called \emph{Polaris} as a deliverable of the Horizon
    2020 OPTICON RadioNet Pilot project.
    This tool has been built on top of the Proposal Data Model, defined in the IVOA standard VO-DML\@.
    Although this is an important part of the tool, this paper will focus on the technologies
    used build and develop \emph{Polaris} itself.

    The API, GUI and CLI are maintained in open-source repositories on GitHub.
    This allows others to contribute to the development of Polaris or fork the projects to change the tool
    to suit their needs.
\end{abstract}

% These lines show examples of subject index entries. At this stage these have to commented
% out, and need to be on separate lines. Eventually, they will be automatically uncommented
% and used to generate entries in the Subject Index at the end of the Proceedings volume.
% Don't leave these in! - replace them with ones relevant to your paper.
%\ssindex{FOOBAR!conference!ADASS 2024}
%\ssindex{FOOBAR!organisations!ASP}

% These lines show examples of ASCL index entries. At this stage these have to commented
% out, and need to be on separate lines. Eventually, they will be automatically uncommented
% and used to generate entries in the ASCL Index at the end of the Proceedings volume.
% The ascl.py command will scan your paper on possible code names.
% Don't leave these in! - replace them with ones relevant to your paper.
%\ooindex{FOOBAR, ascl:1101.010}


%figures are included using the following commands:
%\articlefigure{image.eps}{ref-label}{figure caption}
%\articlefiguretwo{image1.eps}{image2.eps}{ref-label}{figure caption for both}
%\articelfigurefour{1.eps}{2.eps}{3.eps}{4.ps}{ref-label}{figure caption for all}
% probably what a '\clearpage' following figure commands


\section{Motivation}\label{sec:motivation}
\emph{NorthStar} is ``end-of-life''.
\emph{NorthStar} is the name of a current observation proposal tool used both by the radio and optical
astronomy communities.
However, there has been a growing need for a replacement, as it has become too difficult to maintain and develop,
and existing deployments rely on old infrastructure.

The Opticon RadioNet Pilot (ORP) aims to support and develop seamless access to radio and optical,
ground-based astronomy facilities across Europe and the rest of the world.
The ORP attempts to deliver on this aim by developing common standards for observation requests and
specifications, as well as a common framework for data access and processing.
As part of this pilot we are developing a new open source proposal tool that provides
a single access point for the community to create, edit, and submit observation requests to various astronomy
facilities.
The tool also strives to provide a uniform and useful interface for reviewing and allocating proposals by the
time-allocation-committees (TAC) at the relevant astronomy facilities.

We set out to develop this proposal tool with a modern, open source philosophy.
To that end, our code base is currently version controlled in a public repository on
GitHub~\footnote{\url{https://github.com/orppst}}, and makes use of ``GitHub Actions'' to perform continuous
integration testing.
Our intention is to allow the user community to contribute to the projects, either directly by writing and
editing source code, via feedback in terms of bug-fixes and feature requests, or to fork the projects to
better tailor them to their own needs.

As a nod to the outgoing proposal tool, and perhaps signifying progress, we are calling this new proposal tool
\emph{Polaris}.

\section{Architecture}\label{sec:architecture}

Figure~\ref{architectureFigure} shows a diagram representing the architecture of \emph{Polaris}.
The functionality of the tool is exposed as a RESTful Application Programming Interface (API) in a
microservices architecture deployed on Kubernetes.
The API connects with a \emph{Postgres} database using \emph{Hibernate} as an ORM\@.
The database schemas are generated from the ProposalDM, and provide the Java class definitions we can
work with when writing the implementation of the API calls.

\articlefigure[scale=0.4]{P103_architecture.eps}{architectureFigure}{The components that make up the toolkit for \emph{Polaris}}

We have created a web-based Graphical User Interface (GUI) frontend to access our API that has been
written in Typescript using the React framework.
This GUI will be the main access point for those creating, editing, and submitting proposals i.e.,
principal investigators (PI) and co-investigators (CoI).
The GUI accesses the SIMBAD Table Access Protocol (TAP) service as an aid to observational target lookup.

We are also actively developing a Command Line Interface CLI, both as a stand-alone application and
a python library.
The intention of this CLI is to provide convenient access for administrators e.g., TAC members, to
the configurable parts of Polaris, like the operational details of astronomy facilities.

Authorisation to the API is done using KeyCloak and an OpenID Connect (OIDC) server.
You can sign-on to \emph{Polaris} using your orcid ID\@.


\section{Technologies}\label{sec:technologies}

In modern development of ``full stack'' applications, there is an extensive choice in how you set up your
development environment, and how the application is packaged for deployment.

In this section we discuss some of the technologies used to help develop, build, test, package, and deploy
\emph{Polaris} as an observation proposal preparation tool.

\subsection{VO-DML: Virtual Observatory Data Modelling Language}\label{subsec:vodml}

\emph{Polaris} uses the IVOA Proposal Data Model (ProposalDM) draft standard as its native data model.
This standard is built with the International Virtual Observatory Alliance (IVOA) Virtual
Observatory Data Modelling Language (VO-DML).
Using the VO-DML tooling we can generate the JPA data access classes for use in our API\@.
Using this common standard opens up the possibility of being able to exchange proposal information with
other proposal tools if they implement an import mechanism for the model.


\subsection{Quarkus: a Kubernetes-native Java framework}\label{subsec:quarkus}

Quarkus~\footnote{\url{https://quarkus.io/}}, a ``Supersonic Subatomic Java'', is designed around a
container-first philosophy to develop applications with low memory usage and fast startup times.
It is an open source project with a large community of developers and contributors, and a vast ecosystem
of extensions and plugins.

Using Quarkus allows us to concentrate our development effort on the business logic, user experience, and GUI
without having to spend time on much of the underlying boilerplate work.
Combining Quarkus with Kubernetes is relatively seamless and means Polaris can be deployed on any standard
Kubernetes cluster for high resilience and scalability.

Quarkus comes with a built-in development mode that allows developers to update source, resource, and
configuration files that are then automatically reflected in the running application.

\subsection{OpenAPI: code generation}\label{subsec:openapi-code-generation}

OpenApi-codegen\footnote{\url{https://github.com/fabien0102/openapi-codegen}} can automatically generate the functional components and type schemas from
our Java API code to be used in the Typescript GUI\@.
In other words, it creates the ``fetch'' calls to the REST endpoints that form our API\@.
OpenApi-codegen is mostly automatic, but we have found that it requires some manual effort to correct
the generated code every now and again.
However, overall it has saved developer time and effort, especially when making changes to the API that must be
reflected in the GUI\@.

\subsection{Mantine: a React components library}\label{subsec:mantine:-a-react-components-library}

Mantine\footnote{\url{https://mantine.dev/} - named after a Pok\'emon.} is a React components library that
provides many customisable components and hooks, based in TypeScript, to quickly and easily build accessible
frontend web applications.
It is an open source project that, at time of writing, is actively developed and maintained on GitHub.
We have found Mantine to be an intuitive, comprehensive, well documented, and performant library.
We use Mantine with Vite, a fast build tool for web applications, to build our frontend GUI\@.

\section{Future Developments}\label{sec:future-developments}

In general, we are looking to develop and include the following features in \emph{Polaris}:

\begin{itemize}
    \item a centralised database of observatories and their instruments
    \item a single-sign-on service
    \item a ``plug-in'' feature for exposure/sensitivity calculators
    \item access to a spectral line lookup service e.g., \emph{Splatalogue}
\end{itemize}

Please notice that this list is not exhaustive, nor is it written in any particular order, and should be
out-of-date before long.

An active list of ``issues'' is maintained in each project's GitHub repository containing
bug-fix and feature requests.

\section{Summary}\label{sec:summary}
In summary, we have created a full-stack, web-based, open source observation proposal tool, \emph{Polaris},
for intended use within both the radio and optical astronomy communities.
It is our hope that those communities contribute to the development of \emph{Polaris} such that it remains an
active project on GitHub, and continues to mature into the foreseeable future at least.

\emph{Polaris} will replace \emph{NorthStar} as the de-facto observation proposal preparation tool for
the \emph{eMerlin} facility in the near future.

\acknowledgements This work has received funding from the European Union's Horizon 2020 research and innovation
programme, grant agreement No 101004719.
We would also like to acknowledge the individual contributions of Allan Stokes and Michael Ahearn to the
development of our frontend GUI\@.

\bibliography{example}  % For BibTex

% if we have space left, we might add a conference photograph here. Leave commented for now.
% \bookpartphoto[width=1.0\textwidth]{foobar.eps}{FooBar Photo (Photo: Any Photographer)}

\end{document}
