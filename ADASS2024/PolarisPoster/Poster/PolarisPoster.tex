
\documentclass[final]{beamer}

% ====================
% Packages orientation=portrait,
% ====================

\usepackage[T1]{fontenc}
\usepackage{lmodern}
\usepackage[size=custom,width=120,height=213,scale=1.7]{beamerposter}
\usetheme{gemini}
\usecolortheme{uom}
\usepackage{graphicx}
\graphicspath{{./images/}}
\usepackage{booktabs}
\usepackage{tikz}
\usepackage{pgfplots}
\pgfplotsset{compat=1.14}
\usepackage{anyfontsize}

% ====================
% Lengths
% ====================

% If you have N columns, choose \sepwidth and \colwidth such that
% (N+1)*\sepwidth + N*\colwidth = \paperwidth
\newlength{\sepwidth}
\newlength{\colwidth}
\setlength{\sepwidth}{0.025\paperwidth}
\setlength{\colwidth}{0.45\paperwidth}

\newcommand{\separatorcolumn}{\begin{column}{\sepwidth}\end{column}}

% ====================
% Title
% ====================

\title{Polaris: a new open source observation proposal preparation tool}

\author{Darren Walker \and Allan England \and Ben Green \and Paul Harrison}

\institute[shortinst]{JCBA, The University of Manchester, UK}

% ====================
% Footer Logos
% ====================
%logos left. middle, and right in the footer
\logoleftfooter{\includegraphics[height=7cm]{logos/polaris_logo}}
\logomiddlefooter{\includegraphics[height=7cm]{logos/ADASS-XXXIV_Logo_rgb_FullColour_uwWfloC}}
\logorightfooter{\includegraphics[height=7cm]{images/PolarisGitHub}}

\footercontent{
  \small{An Observation Proposal Preparation Tool} \hfill
  ADASS XXXIV Valetta, Malta - 2024 \hfill \hspace{19ex}
    \href{https://github.com/orppst/}{github.com/orppst} \hspace{3ex}
}


% ====================
% Header Logos
% ====================

% use this to include logos on the left and/or right side of the header:
\logoright{\includegraphics[height=7cm]{logos/INSU-Opticon-Radionet-logo-web-blanc-fond-transparent}}
\logoleft{\includegraphics[height=7cm]{logos/UOM_logo_allwhite}}

% ====================
% Figure command
% ====================

\newcommand{\insertFigure}[3][width=1.0\textwidth]{%
    \begin{figure}[ht]
    \centering
    \includegraphics[#1]{#2}
    \caption{#3}
    \label{fig:#2}
    \end{figure}
}


% ====================
% Body
% ====================

\begin{document}

    \begin{frame}[t]
        \begin{columns}[t]
            \separatorcolumn

            \begin{column}{\colwidth}

                \begin{block}{Motivation}

                    \heading{\emph{NorthStar} is dead}
                    Technically ``end-of-life'', but that just doesn't have the same ring to it.
                    \textbf{\emph{NorthStar}} is the name of a current observation proposal preparation tool used both
                    by the radio and optical astronomy communities.
                    It requires an exorbitant amount of effort to maintain and develop, it is awkward to use for both
                    observatories and astronomers, and is in need of replacement.

                    \heading{Opticon RadioNet Pilot}
                    The \textbf{Opticon RadioNet Pilot (ORP)} aims to support and develop seamless access to
                    \textbf{radio and optical}, ground-based astronomy facilities across Europe and the rest of the
                    world.
                    The \textbf{ORP} attempts to deliver on this aim by developing \textbf{common standards for observation
                    requests and specifications}, as well as a common framework for data access and processing.
                    As part of this pilot we are developing \textbf{a new open source proposal tool} that provides a
                    single access point for the community to \textbf{create, edit, and submit proposals} to various
                    astronomy facilities.
                    Additionally, the tool provides an interface for \textbf{reviewing and allocating proposals} by the
                    time-allocation-committees (TAC) at the relevant astronomy facilities.

                    \heading{Polaris}
                    We set out to develop a generalised, ``phase 1'' observation proposal tool with
                    \textbf{a modern, open source philosophy}.
                    To that end, our code base is currently version controlled in a
                    \textbf{private repository on GitHub} (see QR-code, bottom right) that we hope remains active for
                    the foreseeable future.
                    Our intention is to allow both the
                    \textbf{observatory and astronomer communities to contribute to the projects},
                    either directly by \textbf{writing and editing source code}, via feedback in terms of
                    \textbf{bug-fixes and feature requests}, or to \textbf{fork the projects} to better tailor them to
                    their own needs.

                    As a nod to the outgoing proposal tool, and perhaps signifying progress, we are calling this new
                    proposal tool \textbf{\emph{Polaris}} (\emph{NorthStar++} felt a bit too ``on-the-nose'').

                \end{block}

                \begin{block}{Use-cases}

                    \insertFigure[width=38cm,height=40cm]{use-cases}{Actors and high level use cases of Polaris}

                    Figure~\ref{fig:use-cases} shows a diagram representing the actors and high level use-cases of
                    \emph{Polaris}.
                    The following use-cases occur in rough chronological order when configuring and using \emph{Polaris}

                    \begin{itemize}
                        \item \textbf{deploy the system} - special system administrator created, local authorisation
                        \item \textbf{user registers to system} - other uses authorised using the system's mechanisms
                        \item \textbf{create observatory admin} - system admin assigns registered user as observatory admin
                        \item \textbf{configure observatory} - observatory admin adds e.g., telescopes and instruments
                        \item \textbf{create TAC Chair} - observatory admin assigns registered user as TAC chair
                        \item \textbf{add TAC members} - TAC chair may add other users as TAC members
                        \item \textbf{create new proposal cycle} - usually by TAC chair but could be TAC member
                        \item \textbf{create or import whole proposal} - PI can create or import a proposal --or-
                        \item \textbf{clone proposal} - PI can clone a proposal as a seed for a new one
                        \item \textbf{edit observing proposal} - PI or CoI may edit a proposal
                        \item \textbf{submit a proposal} - PI can submit a proposal when it is ready
                        \item \textbf{assign internal reviewer} - TAC chair assigns TAC member from their observatory as a reviewer
                        \item \textbf{assign external reviewer} - TAC chair assigns user external to their observatory as a reviewer
                        \item \textbf{export whole proposal} - Reviewers may export the proposal for offline study
                        \item \textbf{review and score proposals} - Reviewers provide comments and scores for the TAC members to assess
                        \item \textbf{assign observing time} - TAC members assign time to the highest scoring proposals
                        \item \textbf{inform PIs about allocation} - either through Polaris or via email
                    \end{itemize}

                    We have also allowed for the following actions:

                    \begin{itemize}
                        \item PI may want to withdraw a previously submitted proposal if they feel it is no longer valid
                        \item TAC Chair revokes a previously allocated proposal, possibly because the cycle schedule has become over-subscribed
                    \end{itemize}

                \end{block}

            \end{column}

            \separatorcolumn

            \begin{column}{\colwidth}

                \begin{block}{Architecture}

                    \insertFigure[width=30cm,height=35cm]{architecture}{The coloured boxes show microservices that can be run on a Kubernetes cluster}

                    Figure~\ref{fig:architecture} shows a diagram representing the architecture of \emph{Polaris}.
                    The functionality of the tool is exposed as a \textbf{RESTful Application Programming Interface (API)},
                    written in Java, as a \textbf{microservices architecture deployed on Kubernetes}.
                    The API connects with a Postgres database using the Hibernate query language.
                    The database schemas are generated from the Proposal Data Model, and provide the Java class
                    definitions we can work with when writing the implementation of the API calls.

                    We have created a web-based \textbf{Graphical User Interface (GUI)} frontend to access our API
                    that has been written in \textbf{Typescript using the React framework}.
                    This GUI will be the main access point for those creating, editing, and submitting proposals i.e.,
                    principal investigators (PI) and co-investigators (CI).
                    The GUI accesses the \textbf{SIMBAD Table Access Protocol (TAP)} service as an aid to observational
                    target lookup.

                    We are also actively developing a \textbf{Command Line Interface CLI}, also written in Java, to
                    provide a companion access to the API. The intention of this CLI is to provide convenient access
                    for administrators e.g., TAC members, to the configurable parts of Polaris, like the operational
                    details of astronomy facilities.

                    Authorisation to the API is done using \textbf{KeyCloak and an OpenID Connect (OIDC) server}.
                    You can sign-on to \emph{Polaris} using your orcid ID\@.

                \end{block}

                \begin{alertblock}{Import and export your proposals}

                    We know that \emph{Polaris} will not be to every one's taste,
                    ``you can't please all of the people all of the time'', somebody famously once said.
                    \emph{Polaris} is designed to capture succinct details about an observation
                    that will allow observatories to deem if it is a suitable use of their facilities.
                    In this respect it is a so called ``phase 1'' proposal tool.
                    Even so, for some observatories, and indeed for some astronomers, \emph{Polaris} will not
                    capture sufficient details about an observation to effectively assess its suitability.

                    \textbf{\emph{Polaris} is built on top of the International Virtual Observatory Alliance (IVOA)
                        Virtual Observatory Data Modelling Language (VO-DML).
                        As such, we have designed \emph{Polaris} to export and import proposals to and from other,
                        perhaps more comprehensive, proposal tools that use VO-DML as their underpinning data modelling
                        language with minimal manual intervention.
                    }

                \end{alertblock}

                \begin{block}{Technologies used to develop Polaris}

                    Let's face it, software developers are lazy beasts.
                    The less code we have to type the happier, in general, we are.
                    If code can be automatically generated, with little to no human involvement then we are
                    all for it (looking at you ChatGPT).

                    The following technologies where use to help develop, build, test, package, and deploy
                    \emph{Polaris} as an observation proposal preparation tool:

                    \begin{itemize}
                        \item \textbf{Quarkus}: a Kubernetes-native Java framework
                        \item \textbf{OpenAPI-codegen}: generate anything from OpenAPI specs
                        \item \textbf{Mantine}: a React components library
                        \item \textbf{Vite}: a build tool for web applications
                    \end{itemize}

                \end{block}

                \begin{exampleblock}{Future plans}

                    In general, we are looking to develop and include the following features in
                    \textbf{\emph{Polaris}}:

                    \begin{itemize}
                        \item \textbf{a centralised database of observatories and their instruments}
                        \item \textbf{a single-sign-on service}
                        \item \textbf{a ``plug-in'' feature for exposure/sensitivity calculators}
                        \item \textbf{access to a spectral line lookup service e.g., Splatalogue}
                    \end{itemize}

                    Please notice that list is not exhaustive, nor is it written in any particular order,
                    and will, with some luck and much work, be out-of-date before long.

                    An active list of ``issues'' is maintained in each project's GitHub repository containing
                    bug-fix and feature requests, please feel free to browse and contribute to these lists.

                \end{exampleblock}

                \begin{block}{Acknowledgements}
                    \includegraphics[height=\baselineskip]{logos/EU_flag_yellow_eps}
                    This project has received funding from the \textbf{European Union’s Horizon 2020} research and
                    innovation programme under grant agreement \textbf{No 101004719}

                    We would also like to acknowledge the individual contributions to the Polaris frontend
                    GUI from \textbf{Alan Stokes} and \textbf{Michael Ahearn}.
                \end{block}


                %\begin{block}{References}

                 %   \nocite{*}
                 %   \footnotesize{\bibliographystyle{plain}\bibliography{poster}}

                %\end{block}

            \end{column}
            \separatorcolumn
        \end{columns}

    \end{frame}
\end{document}
